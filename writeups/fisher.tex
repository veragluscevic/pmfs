\section{Fisher analysis}
\label{sec:fisher}

We now use the key results of \S\ref{sec:estimators} to discuss estimation of sensitivity of future observations to detecting magnetic fields in the IGM at high redshifts. We first discuss the case of a field uniform in the entire survey volume, starting with the unsatured case, where the strength of $\vec B$ classically produces less than 1 radian of precession at all redshifts of interest, and then move on to discussing detectability in the saturated case, where $B$ is a strong field in this sense. Finally, we discuss detectability of a stochastic magnetic field with a scale-independent power spectrum.

\subsection{Uniform field case}
\label{subsec:uniform_fisher}

If an experiment measures the redshifted 21-cm brightness temperature, and thus provides an estimate of its 2-point statistic, its sensitivity to recovering $B$ is given by the usual Fisher formula \cite{}. For a case of the uniform field, where \eq{\ref{eq:B0}} holds, the sensitivity $\sigma_{B_0}$ to measuring $B_0$ is directly derived from \eq{\ref{eq:B_estimator}},
%\begin{widetext}
\beq
\bga
\sigma_{B_0}^{-2}(z) = \int dV_\mathrm{patch}(z)
\frac{d\vec k}{(2\pi)^3}\left(  \frac{\frac{\partial P^S}{\partial B_0}(\vec k)}{P^N (\vec k)+ P_0^S(\vec k) }\right)^2 
\\
=\int dV_\mathrm{patch}(z)
\frac{k^2dk d\phi_k\sin \theta_kd\theta_k}{2(2\pi)^3}(1+z)^2\\
\times\left( \frac{2P_\delta(k,z)G_0(\theta_k,\phi_k,z)\frac{\partial G_0}{\partial B}(\theta_k, \phi_k,z)}{P^N(k,\theta_k,z) + P_\delta(k,z)G_0^2(\theta_k,\phi_k,z)} \right)^2,
\ega
\label{eq:fisher_patch}
\eeq
%\end{widetext}
where we transitioned from a sum over $\vec k$ modes to an integral, using $\sum_{\vec k} \to V\int d\vec k /(2\pi)^3$. Note that we made dependence on the redshift explicit. Furthermore, the integral is performed over $V_\mathrm{patch}$, the comoving volume of the survey, where
\beq
dV_\mathrm{patch} = \frac{c}{H(z)}D_A^2(z)\Omega_\mathrm{survey}dz,
\label{eq:dVpatch}
\eeq
where $D_A$ is angular diameter distance to $z$.
Note again that we operate under the null assumtion of small $B$, so every summand in the above equation is evaluated for a fiducial case of $B=0$. In this case, the above expression can be used to compute magnitude of $B$ detectable at 1$\sigma$ level for a given noise level (and cosmology). The power spectrum of the 21-cm signal is in the denominator of the summands in order to account for sample variance. A perhaps unusual feature of this analysis is that we do not take the average of the noise over $\phi_k$; this is because the magnetic field introduces anisotropy in the signal.
The integration limits in \eq{\ref{eq:fisher_patch}} are: $\phi_k\in[0,2\pi]$; $\theta_k\in [0,\pi]$; and $k\in[2\pi u_\mathrm{min}/(d_A\sin\theta_k),2\pi u_\mathrm{max}/(d_A\sin\theta_k)]$, where $u_\mathrm{min, max}=\frac{L_\text{min, max}}{\lambda}$ correspond to the maximum and minimum baseline, respectively.

The integral in \eq{\ref{eq:fisher_patch}} is performed on a small, approximately flat, patch of the sky. If the survey area is big enough that this approximation does not hold in the entire survey patch, then this integral should be performed on small parts of the survey, and the results added in quadriture to obrain the sensitivity of the entire survey. This will then account for the change in the angle that a uniform field $\vec B$ makes with a line of sight, as the line of sight moves through a large survey area, and the total sensitivity is given with
\beq
\bga
\sigma^{-2}_{B_0,\text{ survey}} = \frac{\sigma^{-2}_{B_0}}{\Omega_\text{patch}} \int_0^{\theta_\text{survey}}\int_{0}^{2\pi} \cos^2 \theta d\theta d\phi \\
= \frac{\sigma^{-2}_{B_0}\pi}{\Omega_\text{patch}} \left(\theta_\text{survey} + \cos \theta_\text{survey} \sin \theta_\text{survey}\right).
\ega
\label{eq:sigma_sum_survey}
\eeq

%%%   MORE HERE!!!
In the saturated case, we would be unable to measure the exact magnitude of $B$, but a slightly different inquiry becomes interesting. Namely, it is useful to know how sensitive are future experiments to distinguishing staturated case from zero magnetic field. To answer this question, we can write the signal power spectrum as a sum of the contributions from the case $B=0$ and from the saturated case scenario of a very strong field (denoted as infinity), 
\beq
P^S(\vec k) = (1-\xi)P^S(\vec k, B=0) + \xi P^S(\vec k, B\to \infty).
\label{eq:saturated_P}
\eeq
We can then perform the standard Fisher analysis, completely analogous to the unsaturated case, in order to understand constraints on parameter $\xi$. In this case, $\sigma_\xi$ would give a 1$\sigma$ sensitivity to simply detecting a strong magnetic field.

%%%%%%%%%%%%%%%%%%%%%%%%%%%%%%%%%%%%%%%
\subsection{Stochastic field case}
\label{subsec:stochastic_fisher}

Let us now discuss the case of a stochastic field. Using \eq{\ref{eq:NK}}, with a procedure analogous to the case of a uniform field, we get
\begin{widetext}
\beq
\bga
P^N_{B_i}(\vec K) = \left(\frac{1}{2}\int k^2d{k}\sin \theta_kd\theta_kd\phi_k \frac{\left|P_\delta(\vec k')G_0^*(\vec k')\frac{\partial G_0}{\partial B_i}(\vec k') - P_\delta(\vec k)G_0(\vec k)\frac{\partial G_0^*}{\partial B_i}(\vec k)\right|^2}{\left(G^2_0(\vec k)P_\delta(\vec k) + P^N(\vec k)\right)\left(G^2_0(\vec k')P_\delta(\vec k') + P^N(\vec k')\right) } \right)^{-1},
\ega
\label{eq:NK2}
\eeq
\end{widetext}
with the condition $\vec k'=\vec K -\vec k$.

To simplify the calculation, in the following, we only focus on signal-to-noise ratio (SNR) for detecting a particular model of magnetic field power spectrum. Namely, we consider the case where most of the signal comes from the largest modes (smallest $\vec K$'s). In this (squeezed) limit, $\vec K \ll \vec k$ and thus $\vec k \approx \vec k'$, such that the noise power spectrum of \eq{\ref{eq:NK2}} becomes white noise (independent on $\vec K$). If we further denote pixels with Greek indicies, and, as before, retain Roman indicies for components of $\vec B$, then the pixel-noise variance for measuring a single mode $\vec K$ of $B_i$ component is $\sigma_{B_i}^2 \equiv P^N_{B_i}(z)/V_\text{voxel}$, where $V_\text{voxel}$ is volume of a survey voxel (3d pixel). The model for the power spectrum is defined through
\beq
(2\pi)^3\delta_D(\vec K - \vec K') P_{B_iB_j}(\vec K) = \left<B_i^*(\vec K) B_j(\vec K')\right>,
\label{eq:Pbb}
\eeq
which relates to the variance in the transverse component $P_B(\vec K)$ as
\beq
P_{B_iB_j}(\vec K) = (\delta_{ij} - \widehat K_i \widehat K_j) P_B(\vec K),
\label{eq:Pbb_Pb}
\eeq
where $\widehat K_{i/j}$ is a unit vector along the direction of a $B_{i/j}$ component.
In this discussion, as a model example, we consider a scale-independent power spectrum, were
\beq
P_{B}(\vec K) = A_0^2/K^3,
\label{eq:SI}
\eeq
and the amplitude $A_0$ is a free parameter in units of Gauss.

To compute SNR for measuring the amplitude of an arbitrary power-spectrum model in a given redshift slice $z$, we have to perform a sum over all voxels in the survey volume at that $z$. The general expression for SNR is
\beq
\text{SNR}^2 = \frac{1}{2} Tr \left( N^{-1}SN^{-1}S\right),
\label{eq:snr_general}
\eeq
where S is the signal matrix, N is the noise matrix. In our case, these matrices are $3N_\text{voxels}\times 3N_\text{voxels}$ (assuming there are $N_\text{voxels}$ voxels in the entire survey, and that there are 3 components of $\vec B$). In the null case, voxels are independent, and so the noise matrix is diagonal, and the signal is the 3d power spectrum of the vector field $\vec B$. For a single redshift slice, this evaluates to 
\beq
\bga
\text{SNR}^2 (z)= \frac{1}{2} \sum_{i\alpha, j\beta} \frac{S_{i\alpha , j\beta}^2}{P^N_{B_i}(\vec K, z)P^N_{B_j}(\vec K, z)} V_\text{voxel}^2\\=
\frac{1}{2} \sum_{ij} \int d\vec r_\alpha \int d\vec r_\beta \frac{\left< B_i(\vec r_\alpha) B_j(\vec r_\beta)\right>^2}{P^N_{B_i}(\vec K, z)P^N_{B_j}(\vec K, z)},
\ega
\label{eq:snr_z_step1}
\eeq
where $\vec r_{\alpha/\beta}$ represents spatial position of a given voxel, and the expectation value on the rhs of the above equation relates to the model power spectrum. If homogeneity and isotropy are satisfied, the integrand should only depend on the separation vector $\vec s \equiv \vec r_\beta -\vec r_\alpha$, which leads to\footnote{Note that in the last step we used $\int d\vec s |f(\vec s)|^2 = \int \frac{d\vec K}{(2\pi)^3}|\widetilde f(\vec K)|^2$, which hold for an arbitrary function $f$ and its Fourier transform $\widetilde f$.}
\beq  
\bga
\text{SNR}^2 (z) = 
\frac{1}{2} \sum_{ij}  \frac{dV_\text{patch}}{{(P^N_{B_i}(z))^2}}\int d\vec s \left< B_i(\vec r_\beta - \vec s) B_j(\vec r_\beta)\right>^2
\\=
\frac{1}{2} \sum_{ij}  \frac{dV_\text{patch}}{{(P^N_{B_i}(z))^2}} \int d\vec K\left(P_{B_iB_j}(\vec K)\right)^2 (1+z)^4,
\ega
\label{eq:snr_z}
\eeq
where $dV_\text{patch}$ is the volume of a redshift-slice patch at $z$, given by \eq{\ref{eq:dVpatch}}. Note that we use the power spectrum of a comoving field $B$, such that the redshift evolution factors out in the usual way, described in the uniform-field case. Substituting \eq{\ref{eq:SI}}, and integrating over all $z$'s available in the survey, 
\beq
\bga
\text{SNR}^2 =  \frac{A_0^4}{2}  \int_{z_\text{min}}^{z_\text{max}}\frac{dV_\text{patch}}{{(P^N_{B_i}(z))^2}}(1+z)^4 
\int_0^{\pi} \sin\theta d\theta \\
\int_0^{2\pi} d\phi\int_{K_\text{min}(z,\theta,\phi)}^{K_\text{max}(z,\theta,\phi)} \frac{d K}{K^4}\sum_{ij\in \{xx, xy, yx, yy\}}(\delta_{ij} - \widehat K_i\widehat K_j)^2,
\ega
\label{eq:snr_intK}
\eeq
where 
\beq
\widehat K_x = \sin\theta\sin\phi, \text{     }
\widehat K_y = \sin\theta\cos\phi.
\label{eq:hat_K_xy}
\eeq
The sum in the above expression reduces to
\beq
\sum_{ij\in \{xx, xy, yx, yy\}}(\delta_{ij} - \widehat K_i\widehat K_j)^2 = 2 - \sin^2\theta(1+\sin^2\phi\cos^2\phi).
\label{eq:sumij}
\eeq
Substituting this into \eq{\ref{eq:snr_intK}}, we get 
\beq
\bga
\text{SNR}^2 =  \frac{A_0^4}{2} \int_{z_\text{min}}^{z_\text{max}}\frac{dV_\text{patch}}{{(P^N_{B_i}(z))^2}}(1+z)^4 \int_0^{\pi} d\theta\\
\int_0^{2\pi} d\phi(2-\sin^2\theta(1+\sin^2\phi\cos^2\phi)) \int_{K_\text{min}(z,\theta,\phi)}^{K_\text{max}(z,\theta,\phi)} \frac{d K}{K^4}.
\ega
\label{eq:snr_ints}
\eeq
Finally, performing the integration over $K,\theta,\phi$ analitically, for the FFTT case gives
\beq
\text{SNR}^2 =  A_0^4\pi^2\frac{7}{4} \int_{z_\text{min}}^{z_\text{max}}\frac{dV_\text{patch}}{{(P^N_{B_i}(z))^2}}(1+z)^4 \left(\frac{1}{K_\text{min}^3}-\frac{1}{K_\text{max}^3}\right),
\label{eq:snr_ints}
\eeq
where $P^N_{B_i}$ is given by \eq{\ref{eq:NK2}}. When the last expression is evaluated at $A_0=1$, it provides the inverse of a value of $A_0$ detectable at $1\sigma$ sensitivity with a given experiment.