\section{Summary and Conclusions}
\label{sec:conclusions}
%mention lensing here

In Paper I of this series, we proposed a new method to detect extremely weak magnetic fields in the IGM during the Dark Ages, using future 21-cm tomography experiments. In this paper, Paper II, we investigated sensitivity of future radio arrays using this method. We developed minimum-variance-estimator formalism that uses 2-point correlation function of the 21-cm brightness temperature to detect and measure magnetic fields in pre-reionization epoch. 

Our results imply that the next-stage array with a little over a square kilometer of area covered in dipole antennas ina  tightly-packed configuration, observing redshifts from 15 to 35, can in principle reach the sensitivity to detect magnetic fields on the order of $10^{-21}$ Gauss comoving. However, disentangling teh exact spectral shape of a stochastic field is more challenging, and can only be expected in the futuristic scenarios where arrays grow to a size of tens of square kilometers in coverage area. In this analysis, we took into account the noise arising from the presence of the large Galactic foreground signal, but we ignotered more subtle effects such as, for example, frequency dependence of the beams, etc., calculation of which would be necessary to create figures of merit for future experiements. 

While this size of a radio array is still in future development plans for some of the current experiments in terms of the number of antennas (compare to the SKA \cite{2008arXiv0802.1727C}, for example), the number of mode measurements required for this goal corresponds to the computational demands for the next--generation  experiments. 

At the end, we emphasize again that the main limitation to sensitivity of this method to measuring magnetic fields at high redshifts is a mere fact that it is based on a two-scattering process---as soon as quality of the 21-cm statistics reaches the levels necessary to probe second-order processes, the effect we focused on in this series of papers will immediately open up an ``\textit{in situ}'' way to trace miniscule (and possibly primordial) magnetic fields with unprecedented precision. 

  