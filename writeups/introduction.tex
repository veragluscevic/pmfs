\section{Introduction}
\label{sec:intro}

Magnetic fields are ubiquitous in our universe on all observed scales \cite{}. However, the origins of the magnetic fields in Galaxies and on large scales is as of yet an unresolved question \cite{}. Various forms of dynamo mechanisms are proposed to maintain and strengthen magnetic fields \cite{} on large scales, but they typically require seed fields to act \cite{}. The seed fields may be produced in the early stages of structure formation \cite{} through Biermann-battery mechanisms \cite{}, or otherwise may be relics from inflationary times \cite{}. Observations of magnetic fields in high-redshift intergalactic medium (IGM) can thus may provide tools for understanding the origins of these fields in the present-day universe.

Many observational probes have previously been used to search for evidence of magnetic fields at high redshifts \cite{}. Amongst the most sensitive ones are the measurements of the cumulative effect of Faraday rotation in the cosmic-microwave-background polarization maps, which currently place an upper limit of $\sim$$10^{-10}$ Gauss (in comoving units) using the measurements from the Planck sattelite \cite{}. In Paper I of this series \cite{}, we proposed a novel method to measure large-scale magnetic fields during the pre-reionization epoch (the Dark Ages). This method relies on using measurements from future 21-cm brightness-temperature tomography surveys, many of which have pathfinder experiments currently running \cite{}, and plans for the next stage experiments to be realized within the next decade \cite{}. As we show in Paper I, the measurement of statistical anisotropy in the 21-cm signal from the Dark Ages has intrinsic sensitivity to magentic fields in the IGM more than \textit{10 orders of magnitude below the current upper limits}. 

While Paper I layed out the formalism necessary to account for the effect of  magnetic fields on the 21-cm signal, this paper (Paper II of the series) deals with evaluating the sensitivity of future experiments to making a detection. The rest of this paper is organized as follows. In \S\ref{sec:method}, we present a quick overview of the main results in Paper I. In \S\ref{sec:estimators}, we derive minimum-variance estimators for a uniform and a stochastic magnetic field. In \S\ref{sec:fisher}, we set up the Fisher analysis formalism necessary to evaluate detectability. In \S\ref{sec:results}, we present numerical results, and we conclude in \S\ref{sec:conclusions}. Supporting materials (derivation of the variance of visibility) are presented in appendicies.

