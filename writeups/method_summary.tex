\section{Summary of the Method}
\label{sec:method}
\begin{figure}
\centering
\includegraphics[width=.35\textwidth,keepaspectratio=true]{Slide2.pdf}
\includegraphics[width=.35\textwidth,keepaspectratio=true]{Slide3.pdf}
\caption{Illustration of the effect of a magnetic field on hydrogen atoms in the excited state of 21--cm transition at high redshifts. In the classical picture, magnetic moments of the atoms (depicted as red arrows) tend to be aligned with density gradients (upper panel; the gradient is depicted with the background shading), unless they precess about the direction of ambient magnetic field (pointing out of the page on the lower panel). When the precessing atoms decay back into the ground state, the emitted quadrupole (aligned with the direction of the magnetic moments) is misaligned with the incident quadrupole. This offset can be observed as a statistical anisotropy in 21--cm brightness--temperature signal, and used to trace cosmological magnetic fields.\label{fig:precession}}
\end{figure}
Magnetic moments of hydrogen atoms in the excited state of the 21--cm line transition tend to align with the incident quadrupole of the 21--cm radiation from the surrounding medium. This effect of ``ground--state alignment'' \cite{Yan08,Yan12} arises in a cosmological setting due to velocity--field gradients. In the presence of an external magnetic field, the emitted 21--cm quadrupole is misaligned with the incident quadrupole, due to atomic precession (illustrated in Figure \ref{fig:precession}). The resulting emission anisotropy can thus be used to trace magnetic fields at high redshifts.

The main result of Paper I was a calculation of the 21--cm brightness--temperature $T_{\rm }$ fluctuation\footnote{Standard notation, used in other literature and in Paper I of this series, for this quantity is $\delta T_b$; however, we use $T$ here to simplify our expressions.} as a function of the line--of--sight direction ${\bf{\widehat n}}$, in the frame of the emitting ensemble of atoms. The key result there is
\beq
\bga
   T_{\rm }({\bf{\widehat n}}, {\bf{\widehat k}}) = \left( 1 - \frac{T_\gamma}{T_{\rm s}} \right) x_{1{\rm s}} \left( \frac{1+z}{10} \right)^{1/2} \\
  \times \biggl[ 26.4 \ {\rm mK} \Bigl\{ 1 + \left(1 + ({\bf{\widehat k}} \cdot {\bf{\widehat n}})^2 \right)\delta \Bigr\}  
- 0.128 \ {\rm mK} \left( \frac{T_\gamma}{T_{\rm s}} \right)\\ \times x_{1{\rm s}} \left( \frac{1+z}{10} \right)^{1/2}  
 \Bigl\{ 1 + 2 \left(1 + ({\bf{\widehat k}} \cdot {\bf{\widehat n}})^2 \right)\delta \\
- \frac{ \delta }{15} \sum_m \frac{4\pi}{5} \frac{Y_{2 m}({\bf{\widehat k}}) \left[ Y_{2 m} ({\bf{\widehat n}}) \right]^* }{1 + x_{ \alpha, (2) } + x_{{\rm c}, (2)} - i m x_{\rm B}} \Bigr\} \biggr] \mbox{,} 
\ega
\label{eq:tbsoln}
\eeq
where the magnetic field is along the $z$ axis; $x_{\alpha, (2)}$, $x_{{\rm c}, (2)}$ and $x_{\rm B}$ parametrize the rates of depolarization of the ground state by optical pumping and atomic collisions, and the rate of magnetic precession (relative to radiative depolarization), respectively (defined in detail in Paper I). Furthermore,  $T_{\rm s}$ and $T_\gamma$ are the spin temperature and the CMB temperature at redshift $z$, respectively; $\bf{\widehat k}$ is a unit vector in the direction of the wave-vector $\vec k$ of a given density Fourier mode; and $Y_{2 m}$ represent the usual spin--zero spherical harmonics. Figure \ref{fig:hp} illustrates the effect of the magnetic field on the brightness temperature emission pattern in the frame of the atom; shown are quadrupole patterns corresponding to the last term of \eq{\ref{eq:tbsoln}}, for various strengths of the magnetic field. Notice that there is a saturation limit for the field strength---for a strong field, the precession is much faster than the decay of the excited state, and the emission pattern asymptotes to the one shown in the bottom panel of Figure \ref{fig:hp}. Above this limit, as discussed in later Sections, the signal cannot be used to reconstruct the strength of the field; however, in the staturated regime, it is still possible to distinguish presence of a strong magnetic field from the case of no magnetic field, as we discuss in detail in \S\ref{sec:fisher}.
\begin{figure}
\centering
\includegraphics[width=.35\textwidth,keepaspectratio=true]{hp_B_0e+00G.pdf}
\includegraphics[width=.35\textwidth,keepaspectratio=true]{hp_B_1e-18G.pdf}
\includegraphics[width=.35\textwidth,keepaspectratio=true]{hp_B_1e-17G.pdf}
\includegraphics[width=.35\textwidth,keepaspectratio=true]{hp_B_1e-16G.pdf}
\caption{Illustration of the quadrupolar pattern of 21--cm emission from the last ($\vec B$--dependent) term of \eq{\ref{eq:tbsoln}} in the frame of the emitting atoms, for the case where $\vec k$ is perpendicular to ${\bf{\widehat n}}$ (maximal signal), shown in Molleweide projection. Lower panels correspond to increasingly stronger magnetic fields (strength denoted on each panel in comoving units), with the bottom panel corresponding to the saturated case. Notice how the type of quadrupole in the top panel (``weak--field'' regime) is distinct from that in the bottom panel  (``strong--field'' regime). \label{fig:hp}}
\end{figure}

The affect of quadrupole misalignement arises at second order in optical depth (it is a result of a two-scattering process), and is thus a small correction to the total brightness temperature. However, owing to the long lifetime of the excited state (during which even an extremely slow precession has a large cumulative effect on the direction of the quadrupole at second order), the effect of misalignment is exquisitely sensitive to magnetic fields in the IGM at redshifts prior to cosmic reionization. As we showed in Paper I, a minuscule magnetic field of  $10^{-21}$ Gauss (in comoving units) produces order--one changes in the direction of the quadrupole. This implies that a high--precision measurement of the 21--cm brightness--temperature 2--point correlation function intrinsically has that level of sensitivity to magnetic fields in the Dark Ages. We now proceed to develop a formalism to search for this effect with surveys of redshifted 21--cm line, and to identifying experimental setups that can achieve this goal. 